% Vorlesung vom 30.10.2015
\renewcommand{\ldate}{2015-10-30}

\subsection{Krümmung einer ebenen Kurve in Parameterdarstellung}
\includegraphicsdeluxe{krEbKurvInParamd1.jpg}{Krümmung einer ebenen Kurve}{Krümmung einer ebenen Kurve}{fig:krEbKurvInParamd1}

Steigung an der Stelle $ x(t), y(t))$: 
$\tan \alpha = \frac{\dot y (t)}{\dot x (t)}$
$=f'(t)= \frac{df}{dx} (x(t))$ (Abb. \ref{fig:krEbKurvInParamd1})

\paragraph{Beide Seiten nach t ableiten:}
$ \frac{\dot x(t) \ddot y(t) - \dot y(t) \ddot x(t)}{\dot x(t)^2}$
$=\frac{df}{dx \cdot dx} \cdot \dot x(t)$
$\Rightarrow f''(x) = \frac{\dot x(t) \ddot y(t) - \dot y(t) \ddot x(t)}{\dot x(t)^3}$

\paragraph{Krümmung beim Parameterwert t}
$ K(t) = \frac{f''(x)}{\sqrt{1+f'(x)^2}^3}$
$= \frac{\dot x \ddot y - \dot y \ddot x}{\dot x^3 \sqrt{1+\frac{\dot y}{\dot x}^2}^3}$

\paragraph{Krümmung bei t}
$ K(t) = \frac{\ddot y \dot x - \dot y \ddot x}{\rbr{\dot x \sqrt{1+\frac{\dot y}{\dot x}^2}}^3}$
 
\subsection{Beispiel Ellipse}
\includegraphicsdeluxe{bspEllipse1.jpg}{Ellipse}{So kann eine Ellipse ausschauen ;-)}{fig:bspEllipse1}
$ \vektor{x(t) \\ y(t)}$
$=\vektor{a \cos t \\ b \sin t}$

$\dot x = -a \sin t, \ddot x = -a \cos t$\\
$\dot y = b \cos t, \ddot y = -bin \sin t$\\

$K(t) = \frac{b \sin (t) a \sin (t) + b \cos (t) a \cos (t)}{\rbr{- a \sin t \sqrt{1+\frac{b \cos t}{-a \sin t}^2}}^3 }$
$= \frac{a b}{\rbr{- a \sin t \sqrt{1+\frac{a^2 \sin^2 t + b^2 \cos^2 t}{a^2 \sin^2 t} }}^3}$
$= \frac{a b}{\rbr{\pm 1 \sqrt{a^2 \sin^2 t + b^2 \cos^2 t}^3}}$

\paragraph{Krümmung an den Scheiteln ($0\gradi, 90\gradi$)}
$t=0\gradi: K(0) = \frac{a b}{\sqrt{b^2}^3} = \frac{a b }{b^3} = \frac{a}{b^2}$ 
$\Rightarrow \textrm{Krümmungskreisradius} = \frac{b^2}{a}$\\

$t=90\gradi : K(\frac{\pi}{2}) = \frac{a b}{\sqrt{a^2}^3} = \frac{a b}{a^3} = \frac{b}{a^2}$ 
$\Rightarrow \textrm{Krümmungskreisradius} = \frac{a^2}{b}$\\

\includegraphicsdeluxe{senkrDreiecke1.jpg}{Ellipse}{Konstruktion einer Ellipse mit $a=3, b=2$ mit Hilfe von ähnlichen Dreiecken. R und r sind die Krümmungskreismittelpunkte. }{fig:senkrDreiecke1}
Wir suchen nun ähnliche Dreiecke (Abb. \ref{fig:senkrDreiecke1}), bei denen die Seiten Senkrecht zueinander sind. 
$\frac{b}{a} = \frac{a}{R}$
$\Rightarrow bR=a^2 \Rightarrow R = \frac{a^2}{b}$

$\frac{b}{a} = \frac{r}{b} \Rightarrow r=\frac{b^2}{a}$

\section{Kurven in Polarkoordinaten}
\includegraphicsdeluxe{kurvInPolarkoord.jpg}{Kurven in Polarkoordinaten}{Kurven in Polarkoordinaten umrechnen. Gegeben $r(\varphi)$}{fig:kurvInPolarkoord}
Gegeben ist $r(\varphi) \Rightarrow$ übliche Parameterdarstellung: 
$ x(\varphi) = r(\varphi) \cdot \cos \varphi$, 
$ y(\varphi) = r(\varphi) \cdot \sin \varphi$

\subsection{Beispiel Archimedische Spirale}
\includegraphicsdeluxe{ArchimedischeSpirale.png}{Beispiel Archimedische Spirale}{Beispiel Archimedische Spirale (Quelle: Wikipedia)}{fig:ArchimedischeSpirale}
$ r(\varphi) = a \cdot \varphi, a > 0$
$ x(\varphi) = a \cdot \varpi \cdot \cos \varphi$, 
$ y(\varphi) = a \cdot \varpi \cdot \sin \varphi$,
Tangentenvektor: 
$x'(\varphi) = a(1\cos \varphi - \varphi \sin \varphi)$, 
$y'(\varphi) = a(1\sin \varphi + \varphi \cos \varphi)$. \\

Bei $\varphi = 0\gradi$: $(x'(0), y'(0)) = a(1,0)$\\
Bei $\varphi = 2\pi $: $a(\cos 2\pi - 2\pi \sin 2\pi, \sin 2\pi + 2\pi \cos 2\pi) = a(1,2\pi)$

\subsection{Sektorflächeninhalt}
\includegraphicsdeluxe{sektorflaecheninhalt1.jpg}{Sektorflächeninhalt}{Sektorflächeninhalt: Annäherung durch Kreissegment (1), durch eine Funktion (2) und durch die Archimedische Spirale (3).}{fig:sektorflaecheninhalt1}
Annäherung durch Kreissegmente: 
$\sum_{i=1}^{n} \frac{r(\varphi_i)^2 \pi}{2\pi} \DPH $
$=\sum_{i=1}^{n} \frac{1}{2} r(\varphi_i)^2 \DPH $
$\overrightarrow{n\Rightarrow \infty}$
$\int_{\varphi_1}^{\varphi_2} \frac{1}{2} r(\varphi)^2 d\varphi$\\

Archimedische Spirale: $2\pi$\\
$\int_{\varphi_1}^{\varphi_2} \frac{1}{2} r(\varphi)^2 d\varphi $
$= \int_{0}^{2\pi} \frac{1}{2} (a \varphi)^2 d\varphi$
$= \frac{a^2}{2} \int_{0}^{2\pi} \varphi^2 d\varphi$
$= \sbr{ \frac{a^2}{2} \frac{1}{3} \varphi^3}_0^{2\pi}$
$= \frac{a^2}{2} \frac{1}{3} (2\pi)^3$
$=\frac{4}{3} a^2 \pi^3 $ mit $a=1$: 
$ \frac{4}{3} \pi^3 = 41.3$
