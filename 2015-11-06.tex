% Vorlesung vom 06.11.2015
\renewcommand{\ldate}{2015-11-06}

\subsection{Beispiel}
Wir rechnen im $\R^3$ mit dem Vektorfeld $F(x,y,z) = \vektor{0\\0\\-1}$ und der Kurve $K(\varphi) = (\cos \varphi, \sin \varphi, \varphi)$ (Schraubenlinie) \profnote{Bei einer Umdrehung geht es $2\pi$ nach oben}

$\int_{0}^{2\pi} F(K(\varphi)) \cdot K'(\varphi) d\varphi$
$=\int_{0}^{2\pi} \vektor{0\\0\\-1} \cdot \vektor{-\sin \varphi\\ \cos \varphi\\ 1} d\varphi$
$=\int_{0}^{2\pi} -1 d\varphi$
$=-2\pi$ 

\section{Differenzieren von Funktionen des Typs $\R^n \rightarrow \R^m$}
$ f:\R^n \rightarrow \R^m $

$\vektor{x_1\\\vdots \\x_n} \rightarrow \vektor{f_1(x_1,...,x_n)\\\vdots\\f_m(x_1,...,x_n)}$

\subsection{Linearisierung}
$f\vektor{x_1 + \Dx_1\\\vdots\\x_n + \Dx_n\\}$
$=\vektor{f_1(x_1 +\Dx_1, ..., x_n+\Dx_n)\\\vdots\\f_m(x_1 +\Dx_1, ..., x_n+\Dx_n)}$ \profnote{Linearisierung mittels partieller Ableitung}

$\approx \vektor{f_1(x_1, ..., x_n) + \frac{\delta f_1}{\delta x_1} \cdot \Dx_1 + ... + \frac{\delta f_1}{\delta x_n} \cdot \Dx_n \\ \vdots \\ f_m(x_m, ..., x_n) + \frac{\delta f_m}{\delta x_1} \cdot \Dx_1 + ... + \frac{\delta f_m}{\delta x_n} \cdot \Dx_n}$

$= \vektor{f_1(x_1,...,x_n)\\\vdots\\f_m(x_1,...,x_n)} $
$+ \vektor{\frac{\delta f_1}{\delta x_1}, \frac{\delta f_1}{\delta x_2}, ..., \frac{\delta f_1}{\delta x_n}\\ \vdots\\\frac{\delta f_m}{\delta x_1}, \frac{\delta f_m}{\delta x_2}, ..., \frac{\delta f_m}{\delta x_n}}$
$\vektor{\Dx_1\\\Dx_2\\\vdots\\\Dx_n}$ 
\profnote{Die mittlere Matrix nennen wir \textbf{Ableitungsmatrix}, hier: Ableitung von $\R^n \rightarrow \R^m$}

\subsection{Beispiel}
$f : \R^3 \rightarrow \R^3$

$\vektor{x_1\\x_2\\x_3} \rightarrow \vektor{x_1 \cdot x_2^2\\ x_2 \cdot x_3 \\ x_1^2 \cdot x_2 \cdot x_3}$

Ableitungsmatrix: 
$\vektor{x_2^2, 2x_1 x_2, 0\\ 0, x_3, x_2\\2 x_1 x_2 x_3, x_1^2 x_3, x_1^2 x_2}$

Linearisierung: 
$f \vektor{x_1 + \Dx_1\\ x_2 + \Dx_2\\ x_3 + \Dx_3}$
$\approx f\vektor{x_1\\ x_2 \\ x_3	}$
$+\vektor{x_2^2, 2x_1 x_2, 0\\ 0, x_3, x_2\\2 x_1 x_2 x_3, x_1^2 x_3, x_1^2 x_2}$
$\vektor{\Dx_1\\ \Dx_2\\ \Dx_3}$

\subsection{Kettenregel}
% 1 \includegraphicsdeluxe{.jpg}{}{}{fig:}
$ (g\circ f) (x+\Dx) $
$=g(f(x+\Dx))$
$\approx g(\underline{f(x)} + \underline{f'(x) \cdot \Dx})$
$\approx g(f(x)) + g'(f(x)) \cdot [f'(x) \cdot \Dx]$
$= g(f(x)) + \underbrace{[g'(f(x)) \cdot f'(x)]}_{\textrm{Die Ableitungsmatrix von $g\circ f$}} \cdot \Dx$

\paragraph{Kettenregel:}
$(g\circ f)'(x)$
$=\underbrace{g'(f(x))}_{\textrm{Matrix}} \underbrace{\cdot}_{\textrm{Matrizenmultiplikation}} \underbrace{f'(x)}_{\textrm{Matrix}}$

\subsection{Beispiel}
$\R^2 \underrightarrow{f} \R^2 \underrightarrow{g} \R^2$
$f\vektor{x_1\\x_2}$
$=\vektor{x_1 \cdot x_2\\ x_1^2}$
$g\vektor{x_1\\ x_2}$
$=\vektor{2 x_2\\ 3 x_1 x_2}$

Direkt:
$(g\circ f)\vektor{x_1 \\ x_2}$
$= g\vektor{x_1 x_2 \\ x_1^2}$
$= \vektor{2 x_1^2\\ 3 x_1^3 x_2}$

$(g\circ f)'\vektor{x_1 \\ x_2}$
$= \vektor{4 x_1, 0\\ 9 x_1^2 x_2, 3 x_1^3}$

Mit der Kettenregel: 
$f'\vektor{x_1 \\ x_2} $
$=\vektor{x_2, x_1\\ 2 x_1, 0}$,
$g'\vektor{x_1 \\ x_2}$
$=\vektor{0, 2\\3 x_2, 3_x1}$

$g'(f(x)) $
$= g' \vektor{x_1 x_2 \\ x_1^2}$
$= \vektor{0, 2 \\ 3 x_1^2, 3 x_1 x_2}$

$g'(f(x)) \cdot f'(x) $
$=\vektor{0, 2 \\ 3 x_1^2, 3 x_1 x_2}$
$\cdot \vektor{x_2, x_1 \\ 2 x_1, 0}$
$= \vektor{4 x_1, 0 \\ 3 x_1^2 x_2 + 6 x_1^2 x_2, 3 x_1^3}$
$=\vektor{4 x_1, 0 \\ 9 x_1^2 x_2, 3 x_1^3}$

\subsection{Hausaufgabe}
$ f: \R \rightarrow \R^2, x \rightarrow \vektor{x \\ \cos x}$, 
$ g: \R^2 \rightarrow \R, \vektor{x\\y}\rightarrow x^y$

\textbf{gesucht:} $(g\circ f)'$
\begin{enumerate}
\item direkt
\item mit Kettenregel
\end{enumerate}
