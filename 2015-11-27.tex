% Vorlesung vom 27.11.2015
\renewcommand{\ldate}{2015-11-27}

\subsection{Ein System von DGL}
$y_1' = f_1(x_1, y_1, y_2) $, 
$y_2' = f_2(x_1, y_1, y_2) $

Lösung: 
\includegraphicsdeluxe{DGLSysteme1.jpg}{System von DGL}{System von DGL}{fig:DGLSysteme1}
$\varphi_1, \varphi_1$ mit 
$\varphi_1'(x) = f_1(x,\varphi_1(x),\varphi_2(x))$,
$\varphi_2'(x) = f_2(x,\varphi_1(x),\varphi_2(x))$

\textbf{Allgemein:}\\
$y_1' = f_1(x,y_1,y_2,...,y_n)$\\
$y_2' = f_2(x,y_1,y_2,...,y_n)$\\
$\vdots$\\
$y_n' = f_n(x,y_1,y_2,...,y_n)$\\

\textbf{Lösung:} Funktionen $\varphi_1, ..., \varphi_n$ (vgl. Abb. \ref{fig:DGLSysteme1}) mit \\
$\varphi_1'(x) = f_1(x, \varphi_1(x), ..., \varphi_n'(x))$\\ 
$\varphi_2'(x) = f_2(x, \varphi_1(x), ..., \varphi_n'(x))$\\ 
$\vdots$\\
$\varphi_n'(x) = f_n(x, \varphi_1(x), ..., \varphi_n'(x))$\\

\textbf{Die fassen wir nun zusammen:}\\
$ \varphi(x) = \vektor{\varphi_1(x)\\\varphi_2(x)\\\vdots\varphi_n(x)}$ 

Das ist die Parameterdarstellung einer Kurve im $\R^n$. Dabei ist x die Zeit, die Bewegung eines Punktes im $\R^n$. Also ist 

$\varphi'(x) = \vektor{\varphi_1'(x)\\\varphi_2'(x)\\\vdots\varphi_n'(x)}$

der Geschwindigkeitsvektor. 

$f\rbr{x, \vektor{y_1\\\vdots\\y_n}} = \vektor{f_1(x,y_1,...,y_n)\\f_2(x,y_1,...,y_n)\\\vdots\\f_n(x,y_1,...,y_n)\\} = \vektor{y_1'\\\vdots\\y_n'\\}$

f gibt zu jedem Zeitpunkt x und Raumpunkt $y=\vektor{y_1\\\vdots\\y_n}$ einen Vektor aus $\R^n$ an. \\

\textbf{Lösung:} $\varphi(x)$ mit $\varphi'(x) = f(x,\varphi(x))$

\includegraphicsdeluxe{DGLGeschwVekt1.jpg}{DGL und der Geschwindigkeitsvektor}{DGL und der Geschwindigkeitsvektor $\varphi'(x)$}{fig:DGLGeschwVekt1}
Die DGL gibt zu jedem Zeitpunkt x und Raumpunkt $y=\vektor{y_1\\\vdots\\y_n}$ einen Geschwindigkeitsvektor an. Gesucht ist die Bewegung im Raum $\varphi(x)$, die dazu passt (Abb. \ref{fig:DGLGeschwVekt1}).\\

\textbf{Gegeben:} $\varphi(x_0) = y_0$. Falls $\abs{x-x_0}$ klein 
$\Rightarrow$ Näherung: $\varphi(x) = \varphi(x_0) + \varphi'(x_0) \cdot (x-x_0)$
$\varphi(x_0) + f(x_0, \varphi(x_0)) \cdot (x-x_0)$

Auch eine \textit{normale} eindimensionale DGL (Abb. \ref{fig:EindimensionalesDGL1}) kann man so auffassen: 

$y' = f(x,y)$ Lösung $\varphi(x)$

\includegraphicsdeluxe{EindimensionalesDGL1.jpg}{Eindimensionales DGL}{Eindimensionales DGL: Der Punkt y zur Zeit x $\Rightarrow$ Die DGL gibt die Geschwindigkeit vor.}{fig:EindimensionalesDGL1}
$\varphi'(x) = f(x,\varphi(x))$
$\varphi(x)$ fassen wir als Bewegung eines Punktes auf der Geraden $\R$ auf. $\varphi'(x)$ ist dabei die Geschwindigkeit (Abb. \ref{fig:}). Lösung: Bewegung eines Punktes, die dazu passt. 

\subsection{DGL n-ter Ordnung}

\begin{defi}
$ f: \R \times \R^n \rightarrow \R$,
$f(x,y,y',y'',...,y^{(n-1)}) = y^{(n)}$ heißt DGL n-ter Ordnung. Eine Lösung ist $\varphi(x) : I \rightarrow \R$ mit 

* \label{*} $ \varphi^{(n)}(x) = f(x,\varphi(x),\varphi'(x),\varphi''(x),...\varphi^{(n-1)}(x))$
\end{defi}

Das kann man umformen in ein System von DGL (DGL-System)\index{DGL!System}:

$y^{(n)} = f(x,\underbrace{y}_{y_0},\underbrace{y'}_{y_1},...,\underbrace{y^{(n-1)}}_{y_{n-1}} )$

$y'_0 = y_1 = f_1(x,y_0,y_1,...,y_{n-1})$\\
$y'_1 = y_2 = f_2(x,y_0,y_1,...,y_{n-1})$\\
$\vdots$\\
$y'_{n-2} = y_{n-1} = f_{n-1}(x,y_0,y_1,...,y_{n-1})$\\
$y'_{n-1} = f(x,y_0,y_1,...,y_{n-1})$\\
Ist $\varphi(x) = \vektor{\varphi_0(x)\\...\\\varphi_{n-1}(x)\\}$ eine Lösung des Systems, so ist $\varphi_0(x)$ eine Lösung von * in der Definition \ref{*}.

\subsection{Lippschitzbedingung}
\includegraphicsdeluxe{Lippschitzbedingung1.jpg}{Die Lippschitzbedingung}{Die Lippschitzbedingung}{fig:Lippschitzbedingung1}
\begin{defi}[Lippschitzbedingung1]
$f: \R \times \R^n \rightarrow \R^n$, $ (x,y) \rightarrow f(x,y)$ 
genügt der Lippschitzbedingung mit konstanter $L>0$, wenn gilt: 

\color{red}
$\abs{f(x,y) - f(x,\tilde{y})}$
\color{black}
$\leq  L \cdot $
\color{blue}
$\abs{y-\tilde{y}}$ 
\color{black}
(vgl. Abb. \ref{fig:Lippschitzbedingung1})
\end{defi}
