% Vorlesung vom 11.12.2015
\renewcommand{\ldate}{2015-12-11}

\subsection{Runge-Kutta-Verfahren}
\includegraphicsdeluxe{RungeKuttaVerfahren1.jpg}{Runge-Kutta-Verfahren}{Runge-Kutta-Verfahren}{fig:RungeKuttaVerfahren1}
Das Runge-Kutta-Verfahren ist ein Verfahren zur näherungsweisen Lösung von Anfangswertproblemen in der numerischen Mathematik (Abb. \ref{fig:RungeKuttaVerfahren1}).

$ K_1 = f(x_{n}, y_{n})$ mit $y'=f(x,y)$ und Schrittweite h

$ K_2 = f(x_{n} + \frac{h}{2}, y_{n} + \frac{h}{2} \cdot K_1)$

$ K_3 = f(x_{n} + \frac{h}{2}, y_{n} + \frac{h}{2} \cdot K_2)$

$ K_4 = f(x_{n} + h, y_{n} + h \cdot K_3)$

$y_{n+1} = y_n + h (\frac{K_1}{6} + \frac{K_2}{3} + \frac{K_3}{3} + \frac{K_4}{6})$ (gewichtetes Mittel der Steigungen) 

\section{Lineare DGL-Systeme}
Wir haben eine quadratische Matrix A und einen Vektor b. Die Werte von A sind \underline{Funktionen} von x. 

$A = \vektor{a_{11}, ..., a_{1n}\\\vdots\\a_{n1}, ..., a_{nn}}$,
$a_{i,j} : \begin{cases} \R \rightarrow \R\\x\rightarrow a_{i,j}(x) \end{cases}$

$b=\vektor{b_1\\\vdots\\b_n} b_i : \begin{cases} \R \rightarrow \R\\x\rightarrow b_{i}(x) \end{cases}$

$y' = Ay + b$

$\vektor{y_1'\\\vdots\\y_n'} = \vektor{a_{11}, ..., a_{1n}\\\vdots\\a_{n1}, ..., a_{nn}} \vektor{y_1\\\vdots\\y_n} + \vektor{b_1\\\vdots\\b_n} \Rightarrow $ lineares DGL-System

\textbf{Lösung:} $\varphi: \R \rightarrow \R^n$,
$ \varphi = \vektor{\varphi_1\\\vdots\\\varphi_n}$,
$ \varphi' = \vektor{\varphi_1'\\\vdots\\\varphi_n'}$,

$\varphi = A \varphi + b$ ist homogen, wenn b=0.

\begin{satz}
$y' = ay$ sei ein homogenes DGL-System. 
$L_h$ sei die Menge aller Lösungen. 
$L_h = \cbr{\varphi : \R \rightarrow \R^n | \varphi' = A \varphi}$

$L_h$ ist ein n-dimensionaler Vektorraum über $\R$. 

Sind $\varphi_1, ..., \varphi_k$ Lösungen, so sind äquivalent:
\begin{enumerate}
\item $\varphi_1, ..., \varphi_k$ sind linear unabhängig. 
\item Es gibt ein $x_0 \in \R $ mit $\varphi_1(x_0), ..., \varphi_k(x_0)$ sind linear unabhängig ($\varphi_1(x_0)$ sind k Vektoren aus $\R^n $).
\item Für jedes $x\in\R $ sind $\varphi_1(x), ..., \varphi_k(x)$ 
\end{enumerate}
ohne Beweis.
\end{satz}

\begin{defi}
Seien $\varphi_1, ..., \varphi_n$ linear unabhängige Lösungen ($\Rightarrow$ Alle Lösungen, weil es eine Basis ist: 
$ c_1 \varphi_1 + ... + c_n \varphi_n, c_i\in\R$). $\varphi_1, ..., \varphi_n$ heißt Lösungsfundamentalsystem. \\

\textbf{Lösungsmatrix:}

$\Phi = (\varphi_1, ..., \varphi_n)$
$=\rbr{
\begin{array}{cccc}
\varphi_{11} & \varphi_{12} & ... & \varphi_{1n} \\ 
\varphi_{21} & \varphi_{22} & ... & \varphi_{2n} \\ 
\vdots & \vdots & \vdots & \vdots \\ 
\varphi_{n1} & \varphi_{n2} & ... & \varphi_{nn}
\end{array} 
}
$

\textbf{alle Lösungen:}

$\Phi \vektor{c_1\\\vdots\\c_n}$
$=c_1 \varphi_1 + ... c_n \varphi_n$

$\Phi' = (\varphi_1', ..., \varphi_n')$

$\Phi' = A \Phi$

\end{defi}

\subsection{Beispiel}
$ y_1' = -\omega y_2, \omega \in \R$

$ y_1' = \omega y_1$

$\vektor{y_1'\\y_2'}$
$=\vektor{0, -\omega\\\omega,0} \vektor{y_1\\y_2}$

$y' = A\cdot y$

$\varphi_1(x) = \vektor{\cos \omega x\\\sin \omega x},$
$\varphi_2(x) = \vektor{-\sin \omega x\\\cos \omega x}$ sind 2 linear unabhängige Lösungen.

\textbf{Test:} 

$\varphi_1'(x) = \vektor{-\omega \sin \omega x\\\omega \cos \omega x} \checkmark$

$A \cdot \varphi_1 $
$=\vektor{0, -\omega\\\omega,0} \vektor{\cos \omega x\\\sin \omega x} $
$=\vektor{-\omega \sin \omega x\\\omega \cos \omega x} \checkmark$

analog $\varphi_2$

\textbf{Sind $\varphi_1, \varphi_2$ linear unabhängig?}

Setze ein x ein, z.B. $x=0$

$\varphi_2(0) = \vektor{1\\0}, \varphi_2(0) = \vektor{0\\1}$

$\vektor{1\\0}, \vektor{0\\1}$ sind linear unabhängig. 

\textbf{Alle Lösungen: }

$c_1 \varphi_1 + c_2 \varphi_2, c_1, c_2 \in \R$

\begin{satz}
$y' = Ay+b$ sei ein inhomogenes DGL-System. 

Sei $L_h$ die Lösungsmenge des zugehörigen homogenen Systems und $L_I$ ide Lösungsmenge des inhomogenen Systems. 
Für beliebiges $\psi_0 \in L_I$ gilt: $L_I = \psi_0 + L_h$\\

ohne Beweis
\end{satz}

\begin{satz}
$y' = Ay+b$ Sei $\Phi$ ein Lösungsfundamentalsystem für $y' = ay$. Man erhält eine Lösung $\psi(x)$
von $y'Ay + b$ durch den Ansatz $ \psi (x) = \Phi \cdot u(x)$. Dieser Ansatz bzw. dieses Verfahren heißt Variation der Konstanten. $\Phi \cdot c$ ist die Gesamtlösung des homogenen Systems. 

Damit ergibt sich: 

$\Phi(x) \cdot u'(x) = b(x)$

$u'(x) = \Phi^{-1}(x) \cdot b(x)$

$u(x) = \int_{x_0}^{x} \Phi^{-1} \cdot b(t) dt + $ konstanter Vektor.
\end{satz}

\profnote{Das von x lassen wir im Beweis weg.}
\begin{proof}
$\psi = \Phi \cdot u$

$\psi' = \Phi' \cdot u + $
\color{red}
\underline{$\Phi \cdot u'$}
\color{black}
auch: $\psi' = A\psi + b $
$=\underbrace{A \Phi}_{\Phi'} u + b$
$=\Phi' u + $
\color{red}
\underline{$b$}
\color{black}

$\Rightarrow \Phi u' = b$
$\Rightarrow u' = \Phi^{-1} b$
$\Rightarrow u = \int_{x_0}^{x} \Phi^{-1} (t) \cdot b(t) dt + $ konstanter Vektor. 
\end{proof}